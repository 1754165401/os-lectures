%%%%%%%%%%%%%%%%%%%%%%%%%%%%%%%%%%%%%%%%%
% Beamer Presentation
% LaTeX Template
% Version 1.0 (10/11/12)
%
% This template has been downloaded from:
% http://www.LaTeXTemplates.com
%
% License:
% CC BY-NC-SA 3.0 (http://creativecommons.org/licenses/by-nc-sa/3.0/)
%
%%%%%%%%%%%%%%%%%%%%%%%%%%%%%%%%%%%%%%%%%

%----------------------------------------------------------------------------------------
%	PACKAGES AND THEMES
%----------------------------------------------------------------------------------------

\documentclass[UTF8,aspectratio=169,14pt]{ctexbeamer}

\usepackage{hyperref}
\hypersetup{
	colorlinks=true,
	linkcolor=red,
	anchorcolor=blue,
	citecolor=green
}

\mode<presentation> {
	
	% The Beamer class comes with a number of default slide themes
	% which change the colors and layouts of slides. Below this is a list
	% of all the themes, uncomment each in turn to see what they look like.
	
	%\usetheme{default}
	%\usetheme{AnnArbor}
	%\usetheme{Antibes}
	%\usetheme{Bergen}
	%\usetheme{Berkeley}
	%\usetheme{Berlin}
	%\usetheme{Boadilla}
	%\usetheme{CambridgeUS}
	%\usetheme{Copenhagen}
	%\usetheme{Darmstadt}
	%\usetheme{Dresden}
	%\usetheme{Frankfurt}
	%\usetheme{Goettingen}
	%\usetheme{Hannover}
	%\usetheme{Ilmenau}
	%\usetheme{JuanLesPins}
	%\usetheme{Luebeck}
	\usetheme{Madrid}
	%\usetheme{Malmoe}
	%\usetheme{Marburg}
	%\usetheme{Montpellier}
	%\usetheme{PaloAlto}
	%\usetheme{Pittsburgh}
	%\usetheme{Rochester}
	%\usetheme{Singapore}
	%\usetheme{Szeged}
	%\usetheme{Warsaw}
	
	% As well as themes, the Beamer class has a number of color themes
	% for any slide theme. Uncomment each of these in turn to see how it
	% changes the colors of your current slide theme.
	
	%\usecolortheme{albatross}
	%\usecolortheme{beaver}
	%\usecolortheme{beetle}
	%\usecolortheme{crane}
	%\usecolortheme{dolphin}
	%\usecolortheme{dove}
	%\usecolortheme{fly}
	%\usecolortheme{lily}
	%\usecolortheme{orchid}
	%\usecolortheme{rose}
	%\usecolortheme{seagull}
	%\usecolortheme{seahorse}
	%\usecolortheme{whale}
	%\usecolortheme{wolverine}
	
	%\setbeamertemplate{footline} % To remove the footer line in all slides uncomment this line
	%\setbeamertemplate{footline}[page number] % To replace the footer line in all slides with a simple slide count uncomment this line
	
	%\setbeamertemplate{navigation symbols}{} % To remove the navigation symbols from the bottom of all slides uncomment this line
}

\usepackage{graphicx} % Allows including images
\graphicspath{{./figs/}}
\usepackage{booktabs} % Allows the use of \toprule, \midrule and \bottomrule in tables
\usepackage{longtable}
\usepackage{listings}
\usepackage{xcolor}
\lstset{numbers=left, %设置行号位置
	numberstyle=\tiny, %设置行号大小
	keywordstyle=\color{blue}, %设置关键字颜色
	commentstyle=\color[cmyk]{1,0,1,0}, %设置注释颜色
	frame=single, %设置边框格式
	escapeinside=``, %逃逸字符(1左面的键),用于显示中文
	%breaklines, %自动折行
	extendedchars=false, %解决代码跨页时,章节标题,页眉等汉字不显示的问题
	xleftmargin=2em,xrightmargin=2em, aboveskip=1em, %设置边距
	tabsize=4, %设置tab空格数
	showspaces=false %不显示空格
}
% Fonts
% \usepackage{libertine}
% \setmonofont{Courier}
\setCJKsansfont[ItalicFont=Noto Serif CJK SC Black, BoldFont=Noto Sans CJK SC Black]{Noto Sans CJK SC}


%----------------------------------------------------------------------------------------
%	TITLE PAGE
%----------------------------------------------------------------------------------------

\title[第3讲]{第3讲 中断、异常和系统调用} % The short title appears at the bottom of every slide, the full title is only on the title page
\subtitle{第一节:xxxx}
\author{向勇、陈渝} % Your name
\institute[清华大学] % Your institution as it will appear on the bottom of every slide, may be shorthand to save space
{
清华大学计算机系 \\ % Your institution for the title page
\medskip
\textit{xyong,yuchen@tsinghua.edu.cn} % Your email address
}
\date{\today} % Date, can be changed to a custom date

\begin{document}

\begin{frame}
\titlepage % Print the title page as the first slide
\end{frame}
%----------------------------------------------------------------------------------------
\begin{frame}
\frametitle{提纲} % Table of contents slide, comment this block out to remove it
\tableofcontents % Throughout your presentation, if you choose to use \section{} and \subsection{} commands, these will automatically be printed on this slide as an overview of your presentation
\end{frame}
%----------------------------------------------------------------------------------------
%	PRESENTATION SLIDES
%----------------------------------------------------------------------------------------

%------------------------------------------------
\section{第二节:系统调用}% Sections can be created in order to organize your presentation into discrete blocks, all sections and subsections are automatically printed in the table of contents as an overview of the talk
%------------------------------------------------
\begin{frame}
	\frametitle{系统调用}
	\framesubtitle{xxxx}
    \begin{itemize}
        \item 操作系统服务的应用编程接口(API)
	    \begin{itemize}
        	\item 通常由高级语言编写(C、C++和rust等)
        	\item 程序访问通常是通过高层次的API接口而不是直接进行系统调用
    	\end{itemize}
        \item 三种最常用的应用程序编程接口(API)
	    \begin{itemize}
        	\item Win32 API 用于 Windows
        	\item POSIX API 用于 POSIX-based systems (包括UNIX,LINUX, Mac OS X的所有版本)
        	\item Java API 用于JAVA虚拟机(JVM)
    	\end{itemize}
    \end{itemize}
\end{frame}
% 2018-lec3-chy.pdf:P39
%------------------------------------------------
\begin{frame}
	\frametitle{系统调用的实现}
	\framesubtitle{xxxx}
\end{frame}
% 2018-lec3-chy.pdf:P40
%------------------------------------------------
\begin{frame}
	\frametitle{函数调用和系统调用的不同处}
	\framesubtitle{xxxx}
    \begin{columns}
    \begin{column}{0.5\textwidth}
        \begin{figure}
%			\includegraphics[width=0.55\linewidth]{xxxx}
        \end{figure}
    \end{column}
    \begin{column}{0.5\textwidth}
	    \begin{itemize}
		    \item 每个系统调用对应一个系统调用号
		    \begin{itemize}
		    	\item 系统调用接口根据系统调用号来维护表的索引
	    	\end{itemize}
		    \item 系统调用接口调用内核态中的系统调用功能实现,并返回系统调用的状态和结果
		    \item 用户不需要知道系统调用的实现
	 		\begin{itemize}
		    	\item 需要设置调用参数和获取返回结果
		    	\item 操作系统接口的细节大部分都隐藏在应用编程接口后
			    \begin{itemize}
		    		\item 通过运行程序支持的库来管理
	    		\end{itemize}
			\end{itemize}
	    \end{itemize}
    \end{column}
    \end{columns}
\end{frame}
% 2018-lec3-chy.pdf:P41
%------------------------------------------------
\begin{frame}
	\frametitle{函数调用和系统调用的不同处}
%	\framesubtitle{xxxx}
    \begin{columns}
    \begin{column}{0.5\textwidth}
        \begin{figure}
%			\includegraphics[width=0.55\linewidth]{xxxx}
        \end{figure}
    \end{column}
    \begin{column}{0.5\textwidth}
	    \begin{itemize}
	        \item 系统调用
		    \begin{itemize}
	        	\item ECALL和SRET指令用于系统调用
	        	\item 系统调用时,堆栈切换和特权级的转换
	    	\end{itemize}
	        \item 函数调用
		    \begin{itemize}
	        	\item JAL和RET用于常规调用
	        	\item 常规调用时没有堆栈切换
	    	\end{itemize}

	        \item RISC-V手册
		    \begin{itemize}
	        	\item \href{http://crva.ict.ac.cn/documents/RISC-V-Reader-Chinese-v2p1.pdf}{RISC-V-Reader-Chinese-v2p1.pdf}
	    	\end{itemize}
	    \end{itemize}
    \end{column}
    \end{columns}
\end{frame}
%------------------------------------------------
\begin{frame}
	\frametitle{中断、异常和系统调用的开销}
%	\framesubtitle{xxxx}
	\begin{itemize}
        \item 超过函数调用
        \item 开销:
	    \begin{itemize}
	        \item 引导机制
	        \item 建立内核堆栈
	        \item 验证参数
	        \item 内核态映射到用户态的地 址空间
		    \begin{itemize}
	        	\item 更新页面映射权限
	    	\end{itemize}
        	\item 内核态独立地址空间
		    \begin{itemize}
        		\item TLB
	    	\end{itemize}
    	\end{itemize}
	\end{itemize}
\end{frame}
% 2018-lec3-chy.pdf:P42
%----------------------------------------------------------------------------------------
\end{document}
