%%%%%%%%%%%%%%%%%%%%%%%%%%%%%%%%%%%%%%%%%
% Beamer Presentation
% LaTeX Template
% Version 1.0 (10/11/12)
%
% This template has been downloaded from:
% http://www.LaTeXTemplates.com
%
% License:
% CC BY-NC-SA 3.0 (http://creativecommons.org/licenses/by-nc-sa/3.0/)
%
%%%%%%%%%%%%%%%%%%%%%%%%%%%%%%%%%%%%%%%%%

%----------------------------------------------------------------------------------------
%	PACKAGES AND THEMES
%----------------------------------------------------------------------------------------

\documentclass[UTF8,aspectratio=169,14pt]{ctexbeamer}

\usepackage{hyperref}
\hypersetup{
	colorlinks=true,
	linkcolor=red,
	anchorcolor=blue,
	citecolor=green
}

\mode<presentation> {
	
	% The Beamer class comes with a number of default slide themes
	% which change the colors and layouts of slides. Below this is a list
	% of all the themes, uncomment each in turn to see what they look like.
	
	%\usetheme{default}
	%\usetheme{AnnArbor}
	%\usetheme{Antibes}
	%\usetheme{Bergen}
	%\usetheme{Berkeley}
	%\usetheme{Berlin}
	%\usetheme{Boadilla}
	%\usetheme{CambridgeUS}
	%\usetheme{Copenhagen}
	%\usetheme{Darmstadt}
	%\usetheme{Dresden}
	%\usetheme{Frankfurt}
	%\usetheme{Goettingen}
	%\usetheme{Hannover}
	%\usetheme{Ilmenau}
	%\usetheme{JuanLesPins}
	%\usetheme{Luebeck}
	\usetheme{Madrid}
	%\usetheme{Malmoe}
	%\usetheme{Marburg}
	%\usetheme{Montpellier}
	%\usetheme{PaloAlto}
	%\usetheme{Pittsburgh}
	%\usetheme{Rochester}
	%\usetheme{Singapore}
	%\usetheme{Szeged}
	%\usetheme{Warsaw}
	
	% As well as themes, the Beamer class has a number of color themes
	% for any slide theme. Uncomment each of these in turn to see how it
	% changes the colors of your current slide theme.
	
	%\usecolortheme{albatross}
	%\usecolortheme{beaver}
	%\usecolortheme{beetle}
	%\usecolortheme{crane}
	%\usecolortheme{dolphin}
	%\usecolortheme{dove}
	%\usecolortheme{fly}
	%\usecolortheme{lily}
	%\usecolortheme{orchid}
	%\usecolortheme{rose}
	%\usecolortheme{seagull}
	%\usecolortheme{seahorse}
	%\usecolortheme{whale}
	%\usecolortheme{wolverine}
	
	%\setbeamertemplate{footline} % To remove the footer line in all slides uncomment this line
	%\setbeamertemplate{footline}[page number] % To replace the footer line in all slides with a simple slide count uncomment this line
	
	%\setbeamertemplate{navigation symbols}{} % To remove the navigation symbols from the bottom of all slides uncomment this line
}

\usepackage{graphicx} % Allows including images
\graphicspath{{./figs/}}
\usepackage{booktabs} % Allows the use of \toprule, \midrule and \bottomrule in tables
\usepackage{longtable}
\usepackage{listings}
\usepackage{xcolor}
\lstset{numbers=left, %设置行号位置
	numberstyle=\tiny, %设置行号大小
	keywordstyle=\color{blue}, %设置关键字颜色
	commentstyle=\color[cmyk]{1,0,1,0}, %设置注释颜色
	frame=single, %设置边框格式
	escapeinside=``, %逃逸字符(1左面的键),用于显示中文
	%breaklines, %自动折行
	extendedchars=false, %解决代码跨页时,章节标题,页眉等汉字不显示的问题
	xleftmargin=2em,xrightmargin=2em, aboveskip=1em, %设置边距
	tabsize=4, %设置tab空格数
	showspaces=false %不显示空格
}
% Fonts
% \usepackage{libertine}
% \setmonofont{Courier}
\setCJKsansfont[ItalicFont=Noto Serif CJK SC Black, BoldFont=Noto Sans CJK SC Black]{Noto Sans CJK SC}


%----------------------------------------------------------------------------------------
% TITLE PAGE
%----------------------------------------------------------------------------------------

\title[第17讲]{第十七讲 :文件系统} % The short title appears at the bottom of every slide, the full title is only on the title page
\subtitle{第5节:进程文件系统procfs}
\author{向勇、陈渝、李国良} % Your name
\institute[清华大学] % Your institution as it will appear on the bottom of every slide, may be shorthand to save space
{
  清华大学计算机系 \\ % Your institution for the title page
  \medskip
  \textit{xyong,yuchen,liguoliang@tsinghua.edu.cn} % Your email address
}
\date{\today} % Date, can be changed to a custom date

\begin{document}

\begin{frame}
\titlepage % Print the title page as the first slide
\end{frame}
\section{第5节:进程文件系统procfs}
% ### 17.5 procfs
% 
%------------------------------------------------
% \begin{frame}
% \frametitle{提纲} % Table of contents slide, comment this block out to remove it
% \tableofcontents % Throughout your presentation, if you choose to use \section{} and \subsection{} commands, these will automatically be printed on this slide as an overview of your presentation
% \end{frame}
%------------------------------------------------
\begin{frame}[fragile]
    \frametitle{What is \href{https://en.wikipedia.org/wiki/Procfs}{procfs}?}


    \begin{itemize}
        \item \href{http://dtrace.org/blogs/eschrock/2004/06/25/a-brief-history-of-proc/}{Tom Killian} wrote the first implementation of /proc, explained in his paper published in 1984.
	    \begin{itemize}
	        \item It was designed to replace the venerable ptrace system call
      \end{itemize} \pause
        \item It presents information about processes and other \textcolor{blue}{system information}
	    \begin{itemize}
	        \item Provide a more convenient and standardized method for \textcolor{blue}{dynamically accessing} process data held in the kernel
          \item The proc file system acts as an interface to \textcolor{blue}{internal data structures} in the kernel.
          \item The proc filesystem provides a \textcolor{blue}{method of communication} between kernel space and user space.
      \end{itemize} \pause
        \item The proc filesystem (procfs) is a special filesystem
	    \begin{itemize}
	        \item It is mapped to a mount point named /proc at boot time.
          \item All of them have a file size of 0, with the exception of kcore, mtrr  and self.
          \item it as a window into the kernel. it just acts as a pointer to where the actual  process information resides.
      \end{itemize}
    \end{itemize}
% 
\end{frame}
%------------------------------------------------
\begin{frame}[fragile]
    \frametitle{The purpose and contents of procfs}


    \begin{itemize}
        \item /proc/PID/fd
	    \begin{itemize}
	        \item Directory,  which contains all \textcolor{blue}{file descriptors}.
      \end{itemize}
        \item /proc/PID/maps
	    \begin{itemize}
	        \item \textcolor{blue}{Memory  maps} to executables and library files.
      \end{itemize}
        \item /proc/PID/status
	    \begin{itemize}
	        \item \textcolor{blue}{Process  status} in human readable form.
      \end{itemize}
        \item /proc/mtrr
	    \begin{itemize}
	        \item The \href{https://wiki.osdev.org/MTRR}{Memory Type Range Registers} (MTRRs) may be used to control processor  access to memory ranges.
          \item This is most useful when you have a video (VGA)  card on a PCI or AGP bus.
          \item This can increase performance of image write operations 2.5  times or more.
      \end{itemize}
    \end{itemize}
% 
\end{frame}
%------------------------------------------------
\begin{frame}[fragile]
    \frametitle{\href{https://developer.ibm.com/technologies/linux/articles/l-proc/}{Access the Linux kernel using the /proc filesystem}}


    \begin{itemize}
        \item kernel modules
	    \begin{itemize}
	        \item dynamically add or remove code from the Linux kernel.
          \item Inserting, checking, and removing an LKM
          \item Reviewing the kernel output from the LKM
      \end{itemize} \pause
        \item Integrating into the /proc filesystem
	    \begin{itemize}
	        \item To create a virtual file in the /proc filesystem, use the `create\_proc\_entry` function.
          \item To remove a file from /proc, use the `remove\_proc\_entry` function.
          \item Write to a /proc entry (from the user to the kernel) by using a `write\_proc` function.
          \item Read data from a /proc entry (from the kernel to the user) by using the `read\_proc`function.
      \end{itemize}
    \end{itemize}
% 
\end{frame}
%------------------------------------------------
\begin{frame}[fragile]
    \frametitle{Interactive tour of /proc: `ls /proc`}

% 
Reading from and writing to /proc (configuring the kernel): 
% 
\begin{semiverbatim}
cat /proc/sys/net/ipv4/ip\_forward
0
echo "1" > /proc/sys/net/ipv4/ip\_forward
cat /proc/sys/net/ipv4/ip\_forward
1
\end{semiverbatim}
% 
\end{frame}
%------------------------------------------------
\begin{frame}[fragile]
    \frametitle{\href{https://www.cnblogs.com/qiuheng/p/5761877.html}{procfs, sysfs, debugfs in Linux}}


    \begin{itemize}
        \item procfs 历史最早,最初就是用来跟内核交互的唯一方式,用来获取处理器、内存、设备驱动、进程等各种信息。
        \item sysfs 跟 kobject 框架紧密联系,而 kobject 是为设备驱动模型而存在的,所以 sysfs 是为设备驱动服务的。
        \item debugfs 从名字来看就是为debug而生,所以更加灵活。
    \end{itemize}
% 
\end{frame}
%----------------------------------------------
%----------------------------------------------
\end{document}
